\documentclass{beamer}
\usetheme{Boadilla}
\usecolortheme{sidebartab}
\beamertemplatenavigationsymbolsempty
\setbeamertemplate{footline}[frame number]
\usepackage{hyperref} 
\usepackage{graphicx}
\usepackage{color}
\usepackage{booktabs}
\usepackage{listings}
\usepackage{soul}
\usepackage{tikz}
\usepackage[utf8]{inputenc}
\usepackage{CJKutf8}
\usepackage[export]{adjustbox}
\usetikzlibrary{shapes.geometric}

\definecolor{gray}{rgb}{0.4,0.4,0.4}
\definecolor{darkblue}{rgb}{0.0,0.0,0.6}
\definecolor{cyan}{rgb}{0.0,0.6,0.6}
\definecolor{lightgray}{gray}{0.8}

\lstset{
	basicstyle=\ttfamily,
	columns=fullflexible,
	showstringspaces=false,
	commentstyle=\color{gray}\upshape
}

\lstdefinelanguage{XML}
{
	morestring=[b]",
	morestring=[s]{>}{<},
	morecomment=[s]{<?}{?>},
	stringstyle=\color{black},
	identifierstyle=\color{darkblue},
	keywordstyle=\color{cyan},
	morekeywords={xmlns,version,type}% list your attributes here
}

\makeatletter
\newcommand\SoulColor{%
	\let\set@color\beamerorig@set@color
	\let\reset@color\beamerorig@reset@color}
\makeatother

\lstset{language=XML}

\title{XML und RDF: Rückblick und Ausblick}
\author{Markus Stocker}
\date{25. Juni 2018}

\begin{document}

\maketitle

\begin{frame}[plain]{}
	
	\huge
	\begin{center}
		Rückblick
	\end{center}
	
\end{frame}

\begin{frame}{XML}
	
	\begin{itemize}
		\item Extensible Markup Language
		\item Eine erweiterbare, beschreibende \emph{meta} Auszeichnungssprache
		\item Hierarchisch (semi-) strukturierte Daten (Baumstruktur)
	\end{itemize}
	
\end{frame}

\begin{frame}{XML: Wichtige Sprachkonstrukte}
	
	\begin{itemize}
		\item Tag
		\begin{itemize}
			\item Beginnt mit \texttt{<} und endet mit \texttt{>}
			\item Drei Arten: \emph{opening}, \emph{closing}, \emph{empty element}
		\end{itemize}
		\item Element
		\begin{itemize}
			\item Beginnt mit \emph{opening tag} und endet mit entsprechendem \emph{closing tag}
			\item Inhalt entweder Text oder andere Elemente
			\item Beachte Gross-/Kleinschreibung, Sonderzeichen, Verschachtelung
		\end{itemize}
		\item Attribut
		\begin{itemize}
			\item In \emph{opening} oder \emph{empty-element tags}
			\item Innerhalb Klammern \texttt{< >} als \texttt{name="value"} Paare
		\end{itemize}
		\item Dokument
		\begin{itemize}
			\item Enthält ein Wurzelelement
			\item Ist wohlgeformt wenn syntaktisch korrekt
		\end{itemize}
	\end{itemize}
	
\end{frame}

\begin{frame}{XML: Bemerkungen}
	
	\begin{itemize}
		\item XML is mensch- und maschinenlesbar
		\item Tags sind allerdings nur für Menschen Bedeutungsvoll
		\item Verwendet für Datenaustausch, zwischen Anwendungen, im Internet
		\item Interoperabilität beruht allerdings auf Einigung (Schema)
		\item XML kann in Programmiersprachen gelesen/geschrieben werden
		\item XML hat auch Nachteile, insb. Ballast wegen tagging
	\end{itemize}
	
\end{frame}

\begin{frame}{XPath}
	
	\begin{itemize}
		\item Ermöglicht die Verarbeitung von XML Daten
		\item Deklarativer Zugriff auf Teile eines XML Dokuments
		\item Selektion von Elementen und Inhalten
		\item Operationen auf Inhalten
	\end{itemize}
	
\end{frame}

\begin{frame}{XPath: Wichtige Sprachkonstrukte}
	
	\begin{itemize}
		\item Lokalisierungspfad
		\begin{itemize}
			\item Adressierung von Knotenmengen
			\item Setzt sich aus mehreren Einzelschritten zusammen
			\item Diese werden mittels \texttt{/} getrennt
			\item Absolut/relativ und ausführlich/verkürzt
		\end{itemize}
		\item Schritt
		\begin{itemize}
			\item A::KT[P]*
			\item Achse: Navigationsrichtung aus dem Kontextknoten
			\item Knotentest: Gewünschte Knotenmenge
			\item Prädikat: Filterbedingungen
		\end{itemize}
		\item Prädikat
		\begin{itemize}
			\item Filtrierung/Einschänkung der XPath Ergebnissmenge
			\item Definition genauer Zielmenge
			\item Komplexere Problemstellungen
		\end{itemize}
	\end{itemize}
	
\end{frame}

\begin{frame}{DTD}
	
	\begin{itemize}
		\item Sprache zur Spezifikation von XML Dokumente
		\item Beschreibung gültiger Elemente, Attribute, Struktur
		\item Diese werden in einer DTD deklariert
		\item Ermöglicht die Validierung von XML Dokumente
		\item Sprich diese auf Gültigkeit zu prüfen
	\end{itemize}
	
\end{frame}

\begin{frame}{DTD: Deklarationen}
	
	\begin{itemize}
		\item Elemente
		\begin{itemize}
			\item \texttt{<!ELEMENT name inhalt>}
			\item Inhalt: Kindelemente, \texttt{PCDATA}, \texttt{EMPTY}
			\item Inhaltsmodelle: Sequenz, Alternative, Wiederholungen
		\end{itemize}
		\item Attribute
		\begin{itemize}
			\item \texttt{<!ATTLIST element attribut typ wert>}
			\item Typen: \texttt{CDATA}, enumerierte Liste, \texttt{ID}, \texttt{ENTITY}
			\item Werte: Standardwert, \texttt{\#REQUIRED}, \texttt{\#IMPLIED}, \texttt{\#FIXED value}
		\end{itemize}
		\item Entitäten
		\begin{itemize}
			\item \texttt{<!ENTITY name "wert">}
		\end{itemize}
	\end{itemize}
	
\end{frame}

\begin{frame}{XML Schema}
	
	\begin{itemize}
		\item Weitere Sprache zur Spezifikation von XML Dokumente
		\item Adressiert einige Schwächen der DTD
		\item Insb. Datentypen, Namensräume, XML Syntax
		\item Stellt vordefinierte Datentypen zur Verfügung
		\item Definition benutzerdefinierte Datentypen
		\item Ermöglicht detailliertere Spezifikation von XML Dokumente
		\item Strengere Restriktionen die auf Gültigkeit geprüft werden können
	\end{itemize}
	
\end{frame}

\begin{frame}{XML Schema: Datentypen}
	
	\begin{itemize}
		\item Datentypen sind wichtig
		\item \texttt{2018-02-10} und \texttt{0.5} sind Daten von verschiedenem Typ
		\item In XML Schema spezifiziert man diese nicht als \texttt{PCDATA}
		\item Sondern als \texttt{xs:dateTime} und \texttt{xs:decimal}
		\item Zudem spezifiziert \texttt{xs:dateTime} das Format \texttt{YYYY-MM-DD}
		\item \texttt{2018-02-10} ist somit 10. Februar nicht 2. Oktober
	\end{itemize}
	
\end{frame}

\begin{frame}{XML Schema: Deklarationen}
	
	\begin{itemize}
		\item Einfache Elemente (keine Kindelemente oder Attribute)
		\begin{itemize}
			\item \texttt{<xs:element name="..." type="..."/>}
			\item Typen: \texttt{xs:string}, \texttt{xs:boolean}, \texttt{xs:int}, ...
			\item Vorgegebene (\texttt{default}) oder festgelegte (\texttt{fixed}) Werte
			\item Häufigkeit des Elements (\texttt{minOccurs}, \texttt{maxOccurs})
		\end{itemize}
		\item Attribute
		\begin{itemize}
			\item \texttt{<xs:attribute name="..." type="..."/>}
			\item Können vorgegebene oder festgelegte Werte haben
		\end{itemize}
		\item Komplexe Elemente
		\begin{itemize}
			\item \texttt{<xs:element><xs:complexType>...}
			\item Zwei Varianten, eine mit benutzerdefiniertem Datentyp
			\item Ordnung mittels \texttt{xs:all}, \texttt{xs:choice}, \texttt{xs:sequence}
		\end{itemize}
	\end{itemize}
	
\end{frame}

\begin{frame}{Wozu Schema}
	
	\begin{itemize}
		\item Beispiel NASA, ESA und JAXA die Daten austauschen
		\item Einigung über verwendete Tags
		\item Wie auch die Struktur (Elemente, Attribute)
		\item Ziel ist es Unterschiede zu vermeiden
		\item Und so den Datenaustausch zu erleichtern
		\item Die Interoperabilität der Systeme erhöhen
	\end{itemize}
	
\end{frame}

\begin{frame}{Wohlgeformtheit und Gültigkeit}
	
	\begin{itemize}
		\item Ein XML Dokument mit korrekter Syntax ist wohlgeformt
		\item Validiert ein wohlgeformtes Dokument einem Schema ist es gültig
		\item Ein gültiges Dokument ist immer auch wohlgeformt (\emph{well-formed})
		\item Ein wohlgeformtes Dokument ist nicht zwingend gültig (\emph{valid})
	\end{itemize}
	
\end{frame}

\begin{frame}{RDF}
	
	\begin{itemize}
		\item Resource Description Framework
		\item Beschreibung von Ressourcen, digitale, physische, imaginäre Entitäten
		\item Grundlegender Baustein des \emph{Semantic Web}
		\item Formale Spezifikation der Bedeutung von Daten
		\item Graphbasiertes Datenformat, Knoten und gerichtete Kanten
		\item Mittels URI benannt, Literale oder \emph{blank nodes}
	\end{itemize}
	
\end{frame}

\begin{frame}{RDF: Literale und \emph{blank nodes}}
	
	\begin{itemize}
		\item Literale repräsentieren Datenwerte
		\item Können typisiert oder untypisiert sein
		\item Typisierte Literale haben einen Datentyp
		\item Untypisierte Literale können ein \emph{language tag} haben
		\item \emph{Blank nodes} sind unbenannte Ressourcen
		\item Haben strukturelle Funktion für mehrwertige Relationen
	\end{itemize}
	
\end{frame}

\begin{frame}{RDF: Tripel}
	
	\begin{itemize}
		\item Die Elementareinheit in RDF
		\item Eine Struktur die aus drei Elementen besteht
		\item Das Subjekt, das Prädikat und das Objekt
		\item Subjekte und Objekte entsprechen Knoten 
		\item Prädikate entsprechen gerichteten Kanten
		\item Prädikate sind immer benannt (URI)
		\item Subjekte und Objekte können unbenannt sein (\emph{blank node})
		\item Subjekte und Objekte können benannt sein (URI)
		\item Objekte können Literale sein
	\end{itemize}
	
\end{frame}

\begin{frame}{RDF: Syntax}
	
	\begin{itemize}
		\item Tripelmengen können entsprechend einer Syntax serialisiert werden
		\item Besprochene Syntaxen: N-Triples, Turtle, RDF/XML
		\item Jede Syntax hat Vor- und Nachteile
		\item Man kann zwischen Syntaxen convertieren (automatisch)
	\end{itemize}
	
\end{frame}

\begin{frame}{SPARQL}
	
	\begin{itemize}
		\item SPARQL Protocol And RDF Query Language
		\item Abfragesprache für RDF
		\item Ermöglicht deklarativer Zugriff auf RDF Daten
		\item \emph{Tripel pattern} (Tripelmuster) fundamentale Struktur
		\item Besteht aus drei Elementen: Subjekt, Prädikat und Objekt
		\item Wobei diese auch variabel sein können (im Unterschied zum Tripel)
	\end{itemize}
	
\end{frame}

\begin{frame}{SPARQL}
	
	\begin{itemize}
		\item Eine \emph{triple pattern} Menge nennt man \emph{basic graph pattern}
		\item Resultatformate: \texttt{SELECT}, \texttt{CONSTRUCT}, \texttt{ASK}, \texttt{DESCRIBE}
		\item Modifizierer: \texttt{FILTER}, \texttt{ORDER BY}, \texttt{LIMIT}, \texttt{OFFSET}
		\item SPARQL Update um RDF Daten deklarativ zu ändern
		\item SPARQL Endpoints für RDF Datenbank als Web Service
		\item Abfrageoptimierung damit man Antworten möglichst schnell erhält
	\end{itemize}
	
\end{frame}

\begin{frame}{RDF Schema}
	
	\begin{itemize}
		\item RDFS unterstützt das Organisieren von Ressourcen einer Tripelmenge
		\item Grundlegend dabei ist die Gruppierung von Ressourcen
		\item Dafür stellt RDFS das Konstrukt der Klasse zur Verfügung
		\item RDFS ermöglicht Klassen als solche zu definieren
		\item Als Instanzen der Klasse aller Klassen (\texttt{rdfs:Class})
		\item Dabei spielt das Prädikat \texttt{rdf:type} eine wichtige Rolle
	\end{itemize}
	
\end{frame}

\begin{frame}{RDF Schema}
	
	\begin{itemize}
		\item RDFS ermöglich die Definition von Unterklassen (\texttt{rdfs:subClassOf})
		\item Und somit die Erzeugung von Klassenhierarchien
		\item Prädikathierarchien sind ebenfalls möglich (\texttt{rdfs:subPropertyOf})
		\item Klassenzugehörigkeiten (\texttt{rdfs:domain}, \texttt{rdfs:range}) 
	\end{itemize}
	
\end{frame}

\begin{frame}{Ontologien}
	
	\begin{itemize}
		\item Formalisierung von Wissen über eine Domäne
		\item Begriffe und Beziehungen eines Gegenstandsbereiches
		\item Informationsaustausch unter Beihaltung der Bedeutung
		\item Bestandteile: Klassen, Relationen, Instanzen
		\item Aufbau: Schema, Inhalt
		\item Wissensquellen: Mensch, Bücher, Web, Datenbanken
	\end{itemize}
	
\end{frame}

\begin{frame}[plain]{}
	
	\huge
	\begin{center}
		Ausblick
	\end{center}
	
\end{frame}

\begin{frame}{Übersicht}
	
	\begin{itemize}
		\item XLink
		\item XPointer
		\item XSLT
		\item XQuery
		\item Web Ontology Language (OWL)
		\item Regeln
		\item Inferenz
		\item Programme
		\item Ontologien
	\end{itemize}
	
\end{frame}

\begin{frame}[fragile]{XLink}
	
	\begin{itemize}
		\item Ermöglicht das Setzen von Hyperlinks in XML Dokumente
		\item Ähnlich wie \texttt{<a href="">...</a>} in HTML
		\item Allerdings können beliebig benannte Elemente einen XLink setzen
		\item Von Browsern generell nicht unterstützt
	\end{itemize}
	
	\vspace{0.5cm}
	
	\lstset{language=XML}\small
	\begin{lstlisting}
<planets xmlns:xlink="http://www.w3.org/1999/xlink">
  <planet xlink:type="simple" xlink:href="http://planets.org">
    <name>Earth</name>
    <radius>6371</radius>
  </planet>
</planets>
	\end{lstlisting}
	
\end{frame}

\begin{frame}[fragile]{XPointer}
	
	\begin{itemize}
		\item Ermöglicht die Verlinkung auf Teile eines Dokuments
		\item XLink verlink nur auf ganze Dokumente
	\end{itemize}
	
	\vspace{0.5cm}
	
	\lstset{language=XML}\footnotesize
	\begin{lstlisting}
<!-- http://planets.org -->
<planets>
  <planet id="Earth">
    <name>Earth</name>
    <radius>6371</radius>
  </planet>
</planets>

<!-- My list of planets linking to http://planets.org elements -->
<planets xmlns:xlink="http://www.w3.org/1999/xlink">
  <planet xlink:type="simple" xlink:href="http://planets.org#Earth"/>
</planets>
	\end{lstlisting}
	
\end{frame}

\begin{frame}{XSLT}
	
	\begin{itemize}
		\item eXtensible Stylesheet Language Transformations
		\item Styling Sprache für XML
		\item XML Dokumente nach andere XML Dokumente transformieren
		\item Aber auch nach (X)HTML oder andere Formate (z.B. CSV)
		\item XSLT verwendet XPath um XML Dokumente zu navigieren
		\item Elemente und Attribute können hinzugefügt/gelöscht werden
		\item Elemente können anders sortiert werden
		\item Man kann XML Dokumente client- oder serverseitig transformieren
		\item Clientseitig z.B. im Browser mittels JavaScript
		\item Serverseitig z.B. in PHP
		\item XML Daten für jegliche Anwendung zur Präsentation vorbereiten
	\end{itemize}
	
\end{frame}

\begin{frame}[fragile]{XSLT Beispiel: Eingabe Dokument}
	
	\lstset{language=XML}
	\begin{lstlisting}
<planets>
  <planet>
    <name>Earth</name>
    <radius>6371</radius>
  </planet>
  <planet>
    <name>Mars</name>
  </planet>
</planets>
	\end{lstlisting}
	
\end{frame}


\begin{frame}[fragile]{XSLT Beispiel: Transformations Dokument}
	
	\lstset{language=XML}\scriptsize
	\begin{lstlisting}
<xsl:stylesheet version="1.0"
  xmlns:xsl="http://www.w3.org/1999/XSL/Transform">

  <xsl:template match="/">
    <html>
      <body>
        <h2>Solar System Planets</h2>
        <table>
          <tr><th>Name</th><th>Radius</th></tr>
          <xsl:for-each select="planets/planet">
            <tr>
              <td><xsl:value-of select="name"/></td>
              <td><xsl:value-of select="radius"/></td>
            </tr>
          </xsl:for-each>
        </table>
      </body>
    </html>
  </xsl:template>

</xsl:stylesheet>
	\end{lstlisting}
	
\end{frame}

\begin{frame}[fragile]{XSLT Beispiel: Ausgabe Dokument}
	
	\lstset{language=HTML}
	\begin{lstlisting}
<html>
  <body>
    <h2>Solar System Planets</h2>
    <table>
      <tr><th>Name</th><th>Radius</th></tr>
      <tr><td>Earth</td><td>6371</td></tr>
      <tr><td>Mars</td><td></td></tr>
    </table>
  </body>
</html>
	\end{lstlisting}
	
\end{frame}

\begin{frame}[fragile]{XSLT Beispiel: Python}
	
	\lstset{language=Python}
	\begin{lstlisting}
from lxml import etree

xslt = """ MY XSLT DOCUMENT """

xml = """ MY XML DOCUMENT """

transform = etree.XSLT(etree.XML(xslt))

print(str(transform(etree.XML(xml))))
	\end{lstlisting}
	
\end{frame}

\begin{frame}{XQuery}
	
	\begin{itemize}
		\item Abfragesprache für XML
		\item Deklarativer Zugriff auf Elemente und Attribute
		\item Baut auf XPath auf und erinnert an SQL
		\item Kann verwendet werden um XML Daten zu transformieren
	\end{itemize}
	
\end{frame}

\begin{frame}[fragile]{XQuery: Beispiel}
	
	\lstset{language=}
	\begin{lstlisting}
for $x in doc("planets.xml")/planets/planet
where $x/radius>6000
order by $x/radius
return $x/name
	\end{lstlisting}
	
\end{frame}

\begin{frame}{Web Ontology Language (OWL)}
	
	\begin{itemize}
		\item Eine weitere Ontologiesprache
		\item Baut auf RDF Schema auf
		\item Ist wesentlich ausdruckstärker
		\item Man kann damit also komplexeres Wissen formalisieren
		\item Unter Berücksichtigung der Rechenkomplexität
		\item Gleich wie RDFS sind auch OWL Ontologien RDF Dokumente
		\item Seit 2004 eine W3C Recommendation
		\item OWL 2 seit 2012 eine W3C Recommendation
	\end{itemize}
	
\end{frame}

\begin{frame}{Web Ontology Language (OWL)}
	
	\begin{itemize}
		\item Unterklassen und ausserdem äquivalente und disjunkte Klassen
		\item Instanzen können als equivalent oder verschieden deklariert werden
		\item Logische Klassenkonstruktoren ermöglichen komplexe Klassen
		\item Schnitt- (\emph{and}), Vereinigungs- (\emph{or}), Komplementärmenge (\emph{not})
		\item Prädikatbeschränkungen für komplexe Klassenbeschreibungen
		\item Äquivalente, inverse, transitive, symmetrische, funktionelle Prädikate
	\end{itemize}
	
\end{frame}

\begin{frame}{Regeln}
	
	\begin{itemize}
		\item Wissen kann man in Ontologien auch als Regeln erfassen
		\item Schlussfolgerung wenn eine Prämisse erfüllt ist
		\item Beispiel: Eine Frau die ein Kind hat ist eine Mutter
		\item Regeln sind auch Teil einer Ontologie (des Schemas)
		\item Spielen auch eine wichtige Rolle bei Inferenz
	\end{itemize}
	
\end{frame}

\begin{frame}{Inferenz}
	
	\begin{itemize}
		\item Bei der logischen Inferenz geht es um deduktive Schlussfolgerung
		\item RDFS, OWL und Regeln ermöglichen automatische Inferenz
		\item Einfaches Beispiel: Implizite Klasseninstanzen in einer Hierarchie
		\item Weiteres Beispiel: Konsistenzprüfung einer Ontologie
		\item Eines der komplexeren Themen im Bereich Wissensrepräsentation
	\end{itemize}
	
\end{frame}

\begin{frame}{Programme}
	
	\begin{itemize}
		\item Prot{\'e}g{\'e}, TopBraid Composer, OntoStudio
		\item Stardog, Virtuoso, BlazeGraph, AllegroGraph
		\item HermiT, ELK, FaCT++
		\item OWL API, Jena, RDF4J
		\item ...
	\end{itemize}
	
\end{frame}

\begin{frame}{Ontologien}
	
	\begin{itemize}
		\item Es gibt sehr viele existierende Ontologien
		\item Von abstraktem bis sehr konkretem Wissen
		\item Decken viele Themenbereiche
		\item Insbesondere Bioinformatik aber zunehmend auch andere
		\item Repositorien für Ontologien (z.B. EBI Ontology Lookup Service)
		\item Ontologien, Taxonomien, Thesauri
		\item Unterschiedlicher Komplexitäts- und Reifegrad
	\end{itemize}
	
\end{frame}

\end{document}