\documentclass{beamer}
\usetheme{Boadilla}
\usecolortheme{sidebartab}
\beamertemplatenavigationsymbolsempty
\setbeamertemplate{footline}[frame number]
\usepackage{hyperref} 
\usepackage{graphicx}
\usepackage{color}
\usepackage{booktabs}
\usepackage{listings}
\usepackage[utf8]{inputenc}

\definecolor{gray}{rgb}{0.4,0.4,0.4}
\definecolor{darkblue}{rgb}{0.0,0.0,0.6}
\definecolor{cyan}{rgb}{0.0,0.6,0.6}

\lstset{
	basicstyle=\ttfamily,
	columns=fullflexible,
	showstringspaces=false,
	commentstyle=\color{gray}\upshape
}

\lstdefinelanguage{XML}
{
	morestring=[b]",
	morestring=[s]{>}{<},
	morecomment=[s]{<?}{?>},
	stringstyle=\color{black},
	identifierstyle=\color{darkblue},
	keywordstyle=\color{cyan},
	morekeywords={xmlns,version,type}% list your attributes here
}

\title{Einführung in XML}
\author{Markus Stocker}
\date{5. März 2018}

\begin{document}

\maketitle

\begin{frame}
	
	\centering\huge
	Wer hat von XML schon gehört?
	
	\vspace{0.5cm}
	\Large
	Warum das Symbol \texttt{</>}?
	
	\vspace{1cm}
	
	\includegraphics[scale=0.3]{xml.png}
	
	\vspace{0.5cm}
	\tiny
	http://www.iconarchive.com/artist/hopstarter.html 
	
	(CC BY-NC-ND 4.0)
	
\end{frame}

\begin{frame}[fragile]{Was ist XML?}
  
  \begin{itemize}
  	\item XML steht für e\textbf{X}tensible \textbf{M}arkup \textbf{L}anguage
  	\item Auf Deutsch: Erweiterbare Auszeichnungssprache
  	\item Es ist eine \emph{markup language}, eine Auszeichnungssprache
  	\item Eine Sprache zur Auszeichnung von Text in Dokumenten
  	\item Ursprünglich aus typographie, z.B. Farbauszeichnung
  	\item \emph{Markup} ist syntaktisch unterscheidbar von Inhalt
  	\item HTML ist wohl die bekannteste Auszeichnungssprache
  	\item Markup Beispiele
	\begin{lstlisting}
   	  XML: <planet>Earth</planet>
	  HTML: <h2>Earth</h2> 
	  LaTeX: \textbf{Earth}
  	\end{lstlisting}
  \end{itemize}
  
\end{frame}

\begin{frame}{Was is XML?}
	
	\begin{itemize}
		\item XML ist eine \emph{meta} Auszeichnungssprache
		\begin{itemize}
			\item Eine Sprache zur Erstellung anderer Auszeichnungssprachen
			\item XML basierte Auszeichnungssprachen: SVG, XHTML, RSS, ...
		\end{itemize}
		\item XML ist eine \emph{beschreibende} Auszeichnungssprache
		\begin{itemize}
			\item Reine Markierung von Inhalten
			\item Keine Verarbeitungsanweisungen, z.B. Darstellung von Inhalten
		\end{itemize}
	\end{itemize}
	
\end{frame}

\begin{frame}{Was is XML?}
	
  \begin{itemize}
    \item Hierarchisch (semi-) strukturierte Daten
    \item Organisiert als XML Dokument
	\item Ein XML Dokument ist eine Zeichenfolge
	\item Strukturierte Daten als Textdatei
    \item XML ist Menschen und Maschinen lesbar
    \item Texteditor wie Notepad genügt um XML zu lesen und schreiben
    \item XML ist selbstbeschreibend
    \item XML is plattform- und implementationsunabhängig
    \item Standard für viele Office Produkte, z.B. LibreOffice
  \end{itemize}
	
\end{frame}

\lstset{language=XML}

\begin{frame}[fragile]{Beispiel: XML Dokument}
	\begin{lstlisting}
	<planets>
	  <planet>Mercury</planet>
	  <planet>Venus</planet>
	  <planet>Earth</planet>
	  <planet>Mars</planet>
	  <planet>Jupiter</planet>
	  <planet>Saturn</planet>
	  <planet>Uranus</planet>
	  <planet>Neptune</planet>
	  <planet>Pluto</planet>
	</planets>
	\end{lstlisting}	
\end{frame}

\begin{frame}[fragile]{Beispiel: SVG}
	\small
	\begin{lstlisting}
	<?xml version="1.0" encoding="utf-8"?>
	<!DOCTYPE svg PUBLIC "-//W3C//DTD SVG 1.1//EN"
	"http://www.w3.org/Graphics/SVG/1.1/DTD/svg11.dtd">

	<svg version="1.1" 
		id="Layer_1" 
		xmlns="http://www.w3.org/2000/svg" 
		xmlns:xlink="http://www.w3.org/1999/xlink" 
		x="0px" y="0px" width="510px" height="510px" 
		viewBox="0 0 510 510" 
		enable-background="new 0 0 510 510" 
		xml:space="preserve">

		<rect x="5" y="5" fill="none" stroke="#000000" 
			stroke-width="10" stroke-miterlimit="10" 
			width="500" height="500"/>
	</svg>
	\end{lstlisting}	
\end{frame}

\begin{frame}{Warum XML?}
	
	\begin{itemize}
		\item Verbreitete Sprache
		\item Oft benutzt um Daten zwischen Anwendungen auszutauschen
		\item Insbesondere im Datenaustausch über das Internet (Web)
		\item XML Kenntnisse für Softwareentwicklung wichtig
		\item Heute keine Auszeichnung mehr, wird einfach angenommen
	\end{itemize}
	
\end{frame}

\begin{frame}{Lehrveranstaltung BIM-108-01}
	
	\begin{itemize}
		\item Bachelor-Studiengang Informationsmanagement
		\item Pflichtmodul BIM-108---Datenstrukturierung (2. Semester)
		\item Lehrveranstaltung BIM-108-01---Grundlagen XML und RDF
		\item Ansprechperson HSH: Prof. Dr. Christian Wartena
		\item Keine Voraussetzungen 
	\end{itemize}
	
	\vspace{0.5cm}
	
	\url{http://infom.wp.hs-hannover.de/wp-content/uploads/2017/08/StudienhandbuchBIMA5.pdf}
		
\end{frame}

\begin{frame}{Veranstaltungsübersicht}
	
	\scriptsize
	\begin{center}
	\begin{tabular}{rrp{7cm}}
	  & Tag  & Thema \\
	\hline
	1 & 5. März  & Einführung in XML \\
	2 & 12. März & XML: Fortgeschrittene Themen \\
	3 & 19. März & XPath \\
	4 & 26. März & Schema: Document Type Definition (DTD) \\
	5 & 9. April  & Schema: XML Schema \\
	6 & 16. April & Einführung in RDF \\
	7 & 23. April & RDF Syntax: Eine breite Wahl \\
	8 & 30. April & RDF: Fortgeschrittene Themen \\
	9 & 7. Mai  & SPARQL: Die RDF Abfragesprache \\
	10 & 14. Mai & SPARQL: Fortgeschrittene Themen \\
	11 & 28. Mai & Einführung in RDF Schema \\
	12 & 4. Juni & Ontologien mit RDF Schema \\
	13 & 11. Juni & Tools für RDF \\
	14 & 18. Juni & XML und RDF: Rückschau und Ausblick \\
	15 & (?) 25. Juni & Klausur \\
	\hline
	\end{tabular}
	\end{center}
	
\end{frame}

\begin{frame}{Prüfungsform}
	
	\begin{itemize}
		\item Noch nicht festgelegt, mehr Information im April
		\item Voraussichtlich 2-stündige schriftliche Klausur
		\item Zusammengelegt mit BIM-108-02 (Inhaltserschliessung I - Methoden)
		\item Termin voraussichtlich am 25. Juni
	\end{itemize}
	
\end{frame}

\begin{frame}{Literatur}
	
	\begin{itemize}
		\item Margit Becher. XML: DTD, XML-Schema, XPath, XQuery, XSLT, XSL-FO, SAX, DOM. 2009, Springer (Deutsch)	
		\item Pascal Hitzler, Markus Krötzsch, Sebastian Rudolph. Foundations of Semantic Web Technologies. 2010, CRC Press. (Englisch)
		\item Pascal Hitzler, Markus Krötzsch, Sebastian Rudolph, York Sure. Semantic Web: Grundlagen. 2008, Springer. \url{https://doi.org/
			10.1007/978-3-540-33994-6} (Deutsch) 
		\item \url{https://www.w3schools.com}
		\item Weitere online Ressourcen
	\end{itemize}
	
\end{frame}

\begin{frame}{Übersicht}
	
	\begin{itemize}
		\item Geschichte
		\item Sprachkonstrukte
		\item Wohlgeformtheit
	\end{itemize}
	
\end{frame}

\begin{frame}{XML: Geschichte}
	
	\begin{itemize}
		\item Entwicklung GML (Goldfarb, Mosher und Lorie, IBM) in 70er Jahren
		\item Sprache zur Auszeichnung von technischen Dokumenten
		\item Daraus wurde die Standard Generalised Markup Language (ISO, 1986)
		\item Spezifikation zur Definition von Auszeichnungssprachen
		\item HTML wurde die bekannte Anwendung von SGML
		\item Die Verbreitung von HTML führte zu Probleme
		\item Wie z.B. unterschiedliche Interpretation in Browsern
		\item Nicht geeignet für Speicherung und Austausch von Daten
		\item Entwicklung XML, zweite Hälfte 90er Jahren
		\item Seit 1998 eine W3C Recommendation
		\item Wie SGML eine Metasprache aber einfacher zu implementieren
		\item Anders als HTML, strenge Einhaltung der Sprachregeln
	\end{itemize}
	
\end{frame}

\begin{frame}{XML Sprachkonstrukte}
	
	\begin{itemize}
		\item \emph{Comment}
		\item \emph{Tag}
		\item \emph{Element}
		\item \emph{Attribute}
		\item \emph{Document}
		\item \emph{Namespace}
	\end{itemize}
	
\end{frame}

\begin{frame}{Sprachkonstrukte: \emph{Comment}}
	
	\begin{itemize}
		\item Kommentare sind erlaubt
		\item Diese stehen zwischen den Zeichen \texttt{<!--} und \texttt{-->}
		\item Beispiel: \texttt{<!-- Das ist ein Kommentar -->}
		\item Kommentare dürften die Zeichen \texttt{--} nicht enthalten
	\end{itemize}
	
\end{frame}

\begin{frame}{Sprachkonstrukte: \emph{Tag}}
	
	\begin{itemize}
		\item Ein \emph{tag} beginnt mit Zeichen \texttt{<} und endet mit Zeichen \texttt{>}
		\item Beispiel: \texttt{<planet>}
		\item Es gibt drei \emph{tag} Arten
		\begin{itemize}
			\item Das \emph{opening tag}, z.B. \texttt{<planet>}
			\item Das \emph{closing tag}, z.B. \texttt{</planet>}
			\item Das \emph{empty-element tag}, z.B. \texttt{<planet/>}
		\end{itemize}
		\item Beachte \emph{tag} Gross- und Kleinschreibung
			\item Die \emph{tags} \texttt{<planet>} und \texttt{<Planet>} sind nicht gleich
	\end{itemize}
	
\end{frame}

\begin{frame}{Sprachkonstrukte: \emph{Element}}
	
	\begin{itemize}
		\item Ein \emph{element} (Element) ist ein Objekt welches 
		\begin{itemize}
			\item Mit einem \emph{opening tag} beginnt 
			\item Mit einem \textbf{entsprechenden} \emph{closing tag} endet
		\end{itemize}
		\item Beispiel: \texttt{<planet>Earth</planet>}
		\item Bedenke Gross- und Kleinschreibung
		\item \emph{Closing tag} \texttt{</Planet>} entspricht nicht dem \emph{opening tag} \texttt{<planet>}
		\item Ein Element kann leer sein
		\item Beispiel: \texttt{<planet></planet>} oder Kurzform \texttt{<planet/>}
	\end{itemize}
	
\end{frame}

\begin{frame}[fragile]{Sprachkonstrukte: \emph{Element}}
	
	\begin{itemize}
		\item Ein Element beinhaltet Text oder andere Elemente
		\item Beispiele
	\end{itemize}
	
	\begin{lstlisting}
  <!-- Element containing text -->
  <planet>Earth</planet>
	
  <!-- Element containing other elements -->
  <planets>
    <planet>Earth</planet>
    <planet>Mars</planet>
  </planets>
	\end{lstlisting}
	
\end{frame}

\begin{frame}[fragile]{Sprachkonstrukte: \emph{Element}}
	
	\begin{itemize}
		\item Die Zeichen \texttt{<} und \texttt{\&} sind in Text nicht erlaubt
		\item Diese müssen als \texttt{\&lt;} und \texttt{\&amp;} entsprechend kodiert werden
		\item Zudem sollten die Zeichen \texttt{>}, \texttt{'}, und \texttt{"} ebenfalls kodiert werden
		\item Alternative: \texttt{<![CDATA[ ... ]]>}
		\item Beispiele
	\end{itemize}
	
	\begin{lstlisting}
  <planet>Mars radius is &lt; that of Earth</planet>
	
  <planet>Earth is &quot;the only place where 
    life is known to exist&quot;</planet>
	  
  <planet><![CDATA[
    Mars radius is < that of Earth
  ]]></planet>
	\end{lstlisting}
	
\end{frame}

\begin{frame}[fragile]{Sprachkonstrukte: \emph{Element}}
	
	\begin{itemize}
		\item Elemente müssen korrekt verschachtelt werden
		\item Beispiel
	\end{itemize}
	
	\begin{lstlisting}
  <!-- Korrekte Verschachtelung -->
  <planets><planet>Earth</planet></planets>
	
  <!-- Inkorrekte Verschachtelung -->
  <planets><planet>Earth</planets></planet>
	\end{lstlisting}
	
\end{frame}

\begin{frame}[fragile]{Sprachkonstrukte: \emph{Attribute}}
	
	\begin{itemize}
		\item \emph{Opening} und \emph{empty-element tags} können Attribute enthalten
		\item Diese stehen innerhalb der \texttt{<} und \texttt{>} Klammern
		\item Als \texttt{name="value"} Paare 
		\item Wobei \texttt{name} im Element eindeutig sein muss
		\item Attribute enthalten für das Element relevante Daten
		\item Beispiel
	\end{itemize}
	
	\begin{lstlisting}
	<planet name="Earth" radius="6371 km"/>
	\end{lstlisting}
	
\end{frame}

\begin{frame}[fragile]{Sprachkonstrukte: \emph{Element} oder \emph{Attribute}}
	
	\begin{itemize}
		\item Es gibt keine Regeln zur Verwendung von \emph{element} oder \emph{attribute}
		\item Das folgende Beispiel enthält die gleiche Information
		\item Wobei \texttt{name} und \texttt{radius} anders verwendet werden
	\end{itemize}
	
	\begin{lstlisting}
	<planet>
	  <name>Earth</name>
	  <radius>6371 km</radius>
	</planet>
	
	<planet name="Earth" radius="6371 km"/>
	\end{lstlisting}
	
\end{frame}

\begin{frame}[fragile]{Sprachkonstrukte: \emph{Element} oder \emph{Attribute}}
	
	\begin{itemize}
		\item Ein Element kann mehrere Werte enthalten und ist erweiterbar 
	\end{itemize}
	
	\begin{lstlisting}
	<planet>
	  <radius>6371 km</radius>
	</planet>
	
	<planet>
	  <radius>
	    <length>6371</length>
	    <unit>km</unit>
	  </radius>
	</planet>
	\end{lstlisting}
	
\end{frame}

\begin{frame}[fragile]{Sprachkonstrukte: \emph{Document}}
	
	\begin{itemize}
		\item Ein XML \emph{document} (Dokument) ist ein Textdokument
		\item Beginnt mit einer (optionalen) \emph{declaration}
		\item Keine Leerzeichen davor
		\item Versionsnummer und (optional) die Kodierung (meist \texttt{utf-8})
		\item Ein XML Dokument muss genau ein (äusserstes) Element enthalten \item Dieses wird \emph{root element} (Wurzelelement) genannt
		\item Beispiel
	\end{itemize}
	
	\begin{lstlisting}
  <?xml version="1.0" encoding="utf-8"?>
  <planets/>
	\end{lstlisting}
	
\end{frame}

\begin{frame}[fragile]{Sprachkonstrukte: \emph{Namespace}}
	
	\begin{itemize}
		\item Vermeidung von Namenskonflikte
		\item Beispiel: \texttt{<planet>} der Himmelskörper oder die Software?
		\item Diese haben meist unterschiedliche Attribute
		\item Verwendung von Namespräfix
		\item Beispiel: \texttt{<astro:planet>} und \texttt{<soft:planet>}
		\item Bedarf der Definition von \emph{namespaces} (Namensräume)
		\item Namensraum Definition als Attribut in (Wurzel-) Element
		\item Beispiel
	\end{itemize}
	
	\begin{lstlisting}
  <planets>
    <astro:planet xmlns:astro="http://astronomy.org">
      <astro:name>Earth</astro:name>
    </astro:planet>
  </planets>
	\end{lstlisting}
	
\end{frame}

\begin{frame}{Wohlgeformtheit}
	
	\begin{itemize}
		\item Ist ein XML Dokument syntaktisch korrekt ist es wohlgeformt
		\item Wohlgeformtheit (\emph{well-formed}) erwartet, unter anderem
		\begin{itemize}
			\item Genau ein Wurzelelement
			\item Elemente müssen korrekt strukturiert sein
			\item Korrekte Verschachtelung der Elemente
		\end{itemize} 
	\end{itemize}
	
\end{frame}

\begin{frame}{Zusammenfassung}
	
	\begin{itemize}
		\item XML ist eine erweiterbare Auszeichnungssprache
		\item Metasprache zur Spezifikation von Auszeichnungssprachen
		\item Wichtige Sprachkonstrukte: \emph{Tag}, \emph{Element}, \emph{Attribute}, \emph{Document}
		\item Wichtiges Konzept: Wohlgeformtheit
		\item Weit verbreitet, insb. im Datenaustausch (Web)
		\item XML Kenntnisse generell vorausgesetzt in ICT
	\end{itemize}
	
\end{frame}

\end{document}