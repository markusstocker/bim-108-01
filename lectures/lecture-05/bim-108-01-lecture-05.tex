\documentclass{beamer}
\usetheme{Boadilla}
\usecolortheme{sidebartab}
\beamertemplatenavigationsymbolsempty
\setbeamertemplate{footline}[frame number]
\usepackage{hyperref} 
\usepackage{graphicx}
\usepackage{color}
\usepackage{booktabs}
\usepackage{listings}
\usepackage{soul}
\usepackage{tikz}
\usepackage[utf8]{inputenc}
\usepackage{CJKutf8}

\definecolor{gray}{rgb}{0.4,0.4,0.4}
\definecolor{darkblue}{rgb}{0.0,0.0,0.6}
\definecolor{cyan}{rgb}{0.0,0.6,0.6}

\lstset{
	basicstyle=\ttfamily,
	columns=fullflexible,
	showstringspaces=false,
	commentstyle=\color{gray}\upshape
}

\lstdefinelanguage{XML}
{
	morestring=[b]",
	morestring=[s]{>}{<},
	morecomment=[s]{<?}{?>},
	stringstyle=\color{black},
	identifierstyle=\color{darkblue},
	keywordstyle=\color{cyan},
	morekeywords={xmlns,version,type}% list your attributes here
}

\makeatletter
\newcommand\SoulColor{%
	\let\set@color\beamerorig@set@color
	\let\reset@color\beamerorig@reset@color}
\makeatother

\lstset{language=XML}

\title{Schema: XML Schema}
\author{Markus Stocker}
\date{16. April 2018}

\begin{document}

\maketitle

\begin{frame}[fragile]{Rekapitulation}
	
	\begin{itemize}
		\item Wozu brachen wir Schemata?
		\item Was ist die Document Type Definition (DTD)?
		\item Was ist der Unterschied zwischen Wohlgeformtheit und Gültigkeit?
		\item Unterstützt DTD Namensräume?
		\item Gültig oder nicht? Wenn nicht, warum?
	\end{itemize}
	
	\lstset{language=XML}
	\begin{lstlisting}	
  <!ELEMENT planets (planet)>
  <!ELEMENT planet EMPTY>
	
  <planets>
    <planet/>
    <planet/>
  </planets>		
	\end{lstlisting}
	
\end{frame}

\begin{frame}{Übersicht}
	
	\begin{itemize}
		\item Warum DTD nicht genügt
		\item XML Schema
		\item Sprachkonstrukte
		\item Beispiele
	\end{itemize}
	
\end{frame}

\begin{frame}{Wir haben die DTD ...}
	
	\begin{itemize}
		\item Warum brauchen wir eine weitere Sprache für Schema Definition?
		\item DTD hat ein paar Nachteile, und zwar
		\begin{itemize}
			\item Keine Datentypen
			\item Keine Unterstützung für Namensräume
			\item Nicht XML Syntax
		\end{itemize}
		\item XML Schema behebt diese Nachteile
		\item Der Zweck ist gleich: XML Dokumente auf Gültigkeit prüfen
	\end{itemize}
	
\end{frame}

\begin{frame}{XML Schema}
	
	\begin{itemize}
		\item Eine XML basierte Alternative zu DTD
		\item XML Schema ist also in XML geschrieben
		\item Nicht nötig eine weitere Sprache zu lernen
		\item Gleichen Tools, wie Editor, Parser, etc.
		\item Unterstützt Datentypen und stellt vordefinierte zur Verfügung
		\item Ermöglich detailliertere Beschreibung der Daten
		\item Strengere Restriktionen die auf Gültigkeit geprüft werden können
		\item Einfachere Konvertierung zwischen Datentypen
	\end{itemize}
	
\end{frame}

\begin{frame}{XML Schema}
	
	\begin{itemize}
		\item XML Schema ist erweiterbar, weil in XML geschrieben
		\item Ein Schema kann andere Schema verwenden und erweitern
		\item Eigene Datentypen definieren (Benutzerdefiniert)
		\item Diese werden aus den definierten Datentypen abgeleitet
		\item Ein XML Dokument kann auf mehrere XML Schema referenzieren
	\end{itemize}
	
\end{frame}

\begin{frame}{XML Schema}
	
	\begin{itemize}
		\item Unterstützung von Datentypen ist wichtig
		\item In DTD sind die Zeichenketten \texttt{2018-02-10} und \texttt{0.5} \texttt{PCDATA}
		\item Genauer genommen, handelt es sich aber um ein Datum und eine Zahl
		\item Es ist von Vorteil wenn man dies beschreiben kann
		\item Ausserdem, ist das Datum \texttt{2018-02-10} nicht eindeutig
		\item Es könnte der 2. Oktober oder der 10. Februar 2018 sein
		\item Der Datentyp \texttt{xs:date} spezifiziert das Format als \texttt{YYYY-MM-DD}
		\item Somit ist klar, dass \texttt{2018-02-10} der 10. Februar 2018 ist
	\end{itemize}
	
\end{frame}

\begin{frame}[fragile]{Beispiel}
	
	XML Schema
	\lstset{language=XML}
	\begin{lstlisting}	
  <xs:schema xmlns:xs="http://www.w3.org/2001/XMLSchema">
    <xs:element name="name" type="xs:string"/>
  </xs:schema>	
	\end{lstlisting}	

	\vspace{0.5cm}
	
	DTD (zum Vergleich)
	\begin{lstlisting}
  <!ELEMENT name (#PCDATA)>	
	\end{lstlisting}
  
	\vspace{0.5cm}
  
    XML (als Beispiel)
   	\begin{lstlisting}	
  <name>Earth</name>
    \end{lstlisting}
	
\end{frame}

\begin{frame}{Elemente Deklarieren}
	
	\begin{itemize}
		\item Unterscheidung zwischen einfachen und komplexen Elemente
		\item Einfache Elemente sind XML Elemente die Text beinhalten
		\item XML Elemente ohne Kindelemente oder Attribute
		\item Wobei ``Text'' viel sein kann: Zahlen, Datum, Zeichenfolge, ... 
		\item XML Elemente mit Kinder/Attribute sind komplexe Elemente
	\end{itemize}
	
\end{frame}

\begin{frame}[fragile]{Einfache Elemente Deklarieren}
	
	\lstset{language=XML}
	\begin{lstlisting}	
  <xs:element name="..." type="..."/>
    \end{lstlisting}
	
	\begin{itemize}
		\item Der \texttt{name} entspricht dem Elementnamen
		\item Der \texttt{type} entspricht dem Elementtyp
	\end{itemize}
	
\end{frame}

\begin{frame}{Vordefinierte Datentypen (Auswahl)}
	
	\begin{itemize}
		\item \texttt{xs:string}
		\item \texttt{xs:boolean}
		\item \texttt{xs:int}
		\item \texttt{xs:date}
		\item \texttt{xs:time}
		\item \texttt{xs:dateTime}
		\item \texttt{xs:duration}
	\end{itemize}
	
\end{frame}

\begin{frame}[fragile]{Einfache Elemente: Beispiel}
	
	\lstset{language=XML}
	\begin{lstlisting}	
  <xs:element name="name" type="xs:string"/>
  <xs:element name="radius" type="xs:decimal"/>
  
  <name>Earth</name>
  <radius>6371.0</radius>
	\end{lstlisting}
	
\end{frame}

\begin{frame}[fragile]{Einfache Elemente: \emph{Default} und \emph{Fixed} Werte}
	
	\begin{itemize}
		\item Einfache Elemente können vorgegebene oder festgelegte Werte haben
		\item Der vorgegebene Wert wird dem Element automatisch zugewiesen
		\item Angenommen es wird kein anderer Wert angegeben
		\item Der festgelegte Wert wird ebenfalls automatisch zugewiesen
		\item Es ist nicht möglich ein anderer Wert anzugeben
	\end{itemize}
		
	\lstset{language=XML}
	\begin{lstlisting}	
  <xs:element name="name" type="xs:string" fixed="Earth"/>
  <xs:element name="radius" type="xs:decimal" default="0"/>
	\end{lstlisting}
	
\end{frame}

\begin{frame}[fragile]{Einfache Elemente: \emph{minOccurs} und \emph{maxOccurs}}
	
	\begin{itemize}
		\item Spezifikation der Häufigkeit eines Elements
		\item \emph{minOccurs}: Wie oft ein Element mindestens vorkommen muss
		\item \emph{maxOccurs}: Wie oft ein Element höchstens vorkommen kann
		\item Vorgegebener Wert ist 1
		\item \emph{minOccurs} darf nicht grösser sein als \emph{maxOccurs}
		\item Unlimitierte Häufigkeit mit \texttt{maxOccurs="unbounded"}
	\end{itemize}
	
	\lstset{language=XML}
	\footnotesize
	\begin{lstlisting}	
  <xs:element name="name" type="xs:string" maxOccurs="3"/>
  <xs:element name="radius" type="xs:decimal" minOccurs="0"/>
  <xs:element name="name" type="xs:string" minOccurs="2" maxOccurs="10"/>
  <xs:element name="name" type="xs:string" maxOccurs="unbounded"/>
	\end{lstlisting}
	
\end{frame}

\begin{frame}[fragile]{Attribute Deklarieren}
	
	\lstset{language=XML}
	\begin{lstlisting}	
  <xs:attribute name="..." type="..."/>
	\end{lstlisting}
	
	\begin{itemize}
		\item Der \texttt{name} entspricht dem Attributnamen
		\item Der \texttt{type} entspricht dem Attributtyp
		\item Ähnlich wie die Deklaration einfacher Elemente
		\item Einfache Elemente können keine Attribute haben
		\item Attribute selbst werden allerdings als einfache Elemente deklariert
	\end{itemize}
	
\end{frame}

\begin{frame}[fragile]{Attribute: Beispiel}
	
	\lstset{language=XML}
	\begin{lstlisting}	
  <xs:attribute name="radius" type="xs:decimal"/>
	
  <name radius="6371.0">Earth</name>
	\end{lstlisting}
	
\end{frame}

\begin{frame}[fragile]{Attribute: Weitere Eigenschaften}
	
	\begin{itemize}
	\item Attribute können vorgegebene oder festgelegte Werte haben
	\item Attribute sind optional: Mit \texttt{use="required"} sind sie erforderlich
	\end{itemize}
	
\end{frame}

\begin{frame}[fragile]{Komplexe Elemente Deklarieren}
	
	\begin{itemize}
		\item Deklaration von XML Elemente mit Kindelemente und/oder Attribute
		\item Es gibt vier Arten von komplexen Elemente
		\begin{itemize}
			\item Leere Elemente
			\item Elemente die nur andere Elemente enthalten
			\item Elemente die nur Text enthalten
			\item Elemente die Text und andere Elemente enthalten
		\end{itemize}
		\item Können Attribute enthalten
		\item Folgendes Beispiel ist ein komplexes Element
		\item Der Inhalt ist zwar nur Text (einfaches Element)
		\item Allerdings enthält das Element ein Attribut
	\end{itemize}
	
	\lstset{language=XML}
	\begin{lstlisting}	
  <name radius="6371.0">Earth</name>
	\end{lstlisting}
	
\end{frame}

\begin{frame}[fragile]{Komplexe Elemente Deklarieren: Variante I}
	
	\lstset{language=XML}
	\begin{lstlisting}	
  <xs:element name="planet">
    <xs:complexType>
      <xs:sequence>
        <xs:element name="name" type="xs:string"/>
        <xs:element name="radius" type="xs:decimal"/>
      </xs:sequence>
    </xs:complexType>
  </xs:element>
	\end{lstlisting}
	
\end{frame}

\begin{frame}[fragile]{Komplexe Elemente Deklarieren: Variante II}
	
	\lstset{language=XML}
	\begin{lstlisting}	
  <xs:element name="planet" type="PlanetType"/>
	
  <xs:complexType name="PlanetType">
    <xs:sequence>
      <xs:element name="name" type="xs:string"/>
      <xs:element name="radius" type="xs:decimal"/>
    </xs:sequence>
  </xs:complexType>
	\end{lstlisting}
	
\end{frame}

\begin{frame}{Komplexe Elemente Deklarieren: Vergleich}
	
	\begin{itemize}
		\item Variante I entspricht dem XML Dokument und ist einfach
		\item In Variante I kann nur \texttt{planet} die Spezifikation verwenden
		\item In Variante II können mehrere Elemente die Spezifikation verwenden
		\item Andere Elemente mögen ein Name und Radius haben
		\item Spezifiziere den Typ einmal und verwende ihn mehrmals
		\item Weiterer Vorteil der Variante II ist die mögliche Erweiterung
	\end{itemize}
	
\end{frame}

\begin{frame}[fragile]{Komplexe Elemente Erweitern}
	
	\small
	\lstset{language=XML}
	\begin{lstlisting}	
  <xs:element name="ellipsoid" type="EllipsoidType"/>
  <xs:element name="planet" type="PlanetType"/>
	
  <xs:complexType name="EllipsoidType">
    <xs:sequence>
      <xs:element name="radius" type="xs:decimal"/>
    </xs:sequence>
  </xs:complexType>
  
  <xs:complexType name="PlanetType">
    <xs:complexContent>
      <xs:extension base="EllipsoidType">
        <xs:sequence>
          <xs:element name="name" type="xs:string"/>
        </xs:sequence>
      </xs:extension>
    </xs:complexContent>
 </xs:complexType>
	\end{lstlisting}
	
\end{frame}

\begin{frame}[fragile]{Komplexe Leere Elemente}
	
	\lstset{language=XML}
	\begin{lstlisting}	
  <planet name="Earth"/>

  <xs:element name="planet">
    <xs:complexType>
      <xs:attribute name="name" type="xs:string"/>
    </xs:complexType>
  </xs:element>
	\end{lstlisting}
	
\end{frame}

\begin{frame}[fragile]{Komplexe Elemente die nur Text Beinhalten}
	
	\lstset{language=XML}
	\begin{lstlisting}	
  <planet radius="6371.0">Earth</planet>
	
  <xs:element name="planet">
    <xs:complexType>
      <xs:simpleContent>
        <xs:extension base="xs:string">
          <xs:attribute name="radius" type="xs:decimal" />
        </xs:extension>
      </xs:simpleContent>
    </xs:complexType>
  </xs:element>
	\end{lstlisting}
	
\end{frame}

\begin{frame}[fragile]{Komplexe Elemente Ordnen}
	
	\begin{itemize}
		\item Es ist möglich, die Ordnung der Kindelement zu spezifizieren
		\item \texttt{xs:all}: Beliebige Ordnung und jedes Element nur einmal
		\item \texttt{xs:choice}: Ein Element aus der spezifizierten Menge
		\item \texttt{xs:sequence}: Elemente müssen in spezifizierter Ordnung vorkommen
	\end{itemize}
	
\end{frame}


\begin{frame}[fragile]{Einschränkungen (\emph{Restrictions})}
	
	\begin{itemize}
		\item XML Element und Attribut Werte können eingeschränkt werden
		\item Möglich sind Restriktionen auf
		\begin{itemize}
			\item Wertebereich, z.B. \texttt{1000 < radius < 10000}
			\item Wertemenge, z.B. \texttt{Farben = \{Blau, Rot, Grün\}}
			\item Wertemuster, z.B. \texttt{[a-zA-Z]}
			\item Leerzeichen, z.B. \texttt{collapse}
			\item Länge, z.B. Passwort mit Länge 10 Zeichen
		\end{itemize}
	\end{itemize}
	
\end{frame}

\begin{frame}[fragile]{Einschränkungen: Wertebereich}

	\lstset{language=XML}
	\begin{lstlisting}		
  <xs:element name="radius">
    <xs:simpleType>
      <xs:restriction base="xs:integer">
        <xs:minInclusive value="1000"/>
        <xs:maxInclusive value="10000"/>
      </xs:restriction>
    </xs:simpleType>
  </xs:element>
	\end{lstlisting}
	
\end{frame}

\begin{frame}[fragile]{Einschränkungen: Wertemenge}

	\lstset{language=XML}
	\begin{lstlisting}	
  <xs:element name="planet">
    <xs:simpleType>
      <xs:restriction base="xs:string">
        <xs:enumeration value="Venus"/>
        <xs:enumeration value="Earth"/>
        <xs:enumeration value="Mars"/>
      </xs:restriction>
    </xs:simpleType>
  </xs:element>
	\end{lstlisting}
		
\end{frame}

\begin{frame}[fragile]{Einschränkungen: Wertemuster}
	
	\lstset{language=XML}
	\begin{lstlisting}	
  <xs:element name="name">
    <xs:simpleType>
      <xs:restriction base="xs:string">
        <xs:pattern value="[A-Z][a-z]+"/>
      </xs:restriction>
    </xs:simpleType>
  </xs:element>
	\end{lstlisting}
	
\end{frame}

\begin{frame}[fragile]{Namesräume}
	
	\lstset{language=XML}
	\begin{lstlisting}	
  <xs:schema xmlns:xs="http://www.w3.org/2001/XMLSchema"
    targetNamespace="http://example.org">

    <xs:element name="planet">
      <xs:complexType>
        <xs:sequence>
          <xs:element name="name" type="xs:string"/>
          <xs:element name="radius" type="xs:decimal"/>
        </xs:sequence>
      </xs:complexType>
    </xs:element>
  
  </xs:schema>
	\end{lstlisting}
	
\end{frame}

\begin{frame}{Zusammenfassung}
	
	\begin{itemize}
		\item XML Schema behebt einige Nachteile der DTD, isb. Datentypen
		\item Gleicher Zweck, XML Dokumente auf Gültigkeit prüfen
		\item Grundidee ist benötigte Elemente und Attribute zu deklarieren
		\item XML Schema und DTD ermöglichen und erhöhen Interoperabilität
	\end{itemize}
	
\end{frame}

\end{document}